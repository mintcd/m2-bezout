\section{Exercise 1.} Describe in detail a Turing machine that decides the language
$$\L=\{w\in \{0,1\}^* \,|\, w \text{ contains equally 0s and 1s}\}.$$

We use the convention that the head points to the symbol right after the start symbol $\triangleright$. We need another symbol $2$ in our alphabet. The idea is that when the string is on the tape, we will traverse it multiple times and at each time, we modify a pair of $0$ and $1$ to $2$ and $2$. A string is in $\L$ if it can be modified to a string of all $2$'s.
\begin{itemize}
  \item At the state $q_\text{init}$, if the tape contains all $2$'s followed by blanks, the original input is accepted. If $0$ is read, the machine goes to the state $q_1$. If $1$ is read, the machine goes to the state $q_2$.
  \item At the state $q_1$, we wait for a $1$, so we loop around when encountering a $0$ or $2$. If a blank is encountered before a $1$ can be read, the string is rejected. If there is a $1$, we move to state $q_3$. Similarly, $q_2$ is the state where we intend to wait for a $0$.
  \item At the state $q_3$, we move to the left until encountering $\triangleright$, by which we move back to $q_\text{init}$ and start reading the next symbol.
\end{itemize}

\begin{figure}[ht]
  \centering
  \begin{tikzpicture}[shorten >=1pt,node distance=2cm,on grid,auto]

    \node (init) {};
    \node[state] (q_init) [below=1cm of init] {$q_\text{init}$};
    \node[state] (q1) [below=4cm of q_init, xshift=-4cm] {$q_1$};
    \node[state] (q2) [below=4cm of q_init, xshift=4cm] {$q_2$};
    \node[state] (q_accept) [below=4cm of q_init] {$q_\text{accept}$};
    \node[state] (q_reject) [below=2cm of q_accept] {$q_\text{reject}$};
    \node[state] (q3) [below=4cm of q_accept] {$q_3$};

    \path[->]
    (init) edge node {} (q_init)
    (q_init) edge node[sloped, above] {$0 \to 2, R$} (q1)
    (q_init) edge node[sloped, above] {$1 \to 2, R$} (q2)
    (q_init) edge node[sloped, above] {$\sqcup \to \sqcup, R$} (q_accept)

    (q_init) edge[loop, out=-15, in=15, distance=1cm] node[anchor=west] {$2\to 2, R$} (q_init)

    (q1) edge[loop, out=165, in=195, distance=1cm] node[anchor=east] {\shortstack{$0 \to 0, R$ \\ $2 \to 2, R$}}
    (q1)
    (q1) edge node[sloped, above] {$\sqcup \to \sqcup, R$} (q_reject)
    (q1) edge node[sloped, above] {$1 \to 2, L$} (q3)

    (q2) edge[loop, out=-15, in=15, distance=1cm] node[anchor=west] {\shortstack{$1 \to 1, R$ \\ $2 \to 2, R$}}
    (q2)
    (q2) edge node[sloped, above] {$\sqcup \to \sqcup, R$} (q_reject)
    (q2) edge node[sloped, above] {$0 \to 2, L$} (q3)


    (q3) edge[loop, out=-105, in=-75, distance=1cm] node[anchor=north] {$\shortstack{$0 \to 0, L$ \\ $1 \to 1, L$ \\ $2 \to 2, L$}$} (q3);

    \draw[->, rounded corners=5pt]
    (q3) -- ($(q3)+(-7cm,0)$)
    -- node[sloped, above] {$\triangleright\to\triangleright, R$} ($(q_init)+(-7cm,0)$)
    -- (q_init);

    % (q3) edge[out=180, in=180, distance=9cm] node[sloped, below] {$\triangleright\to\triangleright, R$} (q_init)
    ;
  \end{tikzpicture}
\end{figure}

\section{Exercise 2.} Describe a TM that decides the language $\L=\{w\#w \,|\, w\in\{0,1\}^*\}$.

The idea is similar to Exercise 1. The difference is that if $0$ is encountered, we \textit{wait for another $0$ only after $\#$}. The case of encountering $1$ is the same.

\begin{figure}[ht]
  \centering
  \begin{tikzpicture}[shorten >=1pt,node distance=2cm,on grid,auto]

    \node (init) {};
    \node[state] (q_init) [below=1cm of init] {$q_\text{init}$};
    \node[state] (q1) [below=4cm of q_init, xshift=-4cm] {$q_1$};
    \node[state] (q1') [below=4cm of q1] {$q_1'$};
    \node[state] (q2) [below=4cm of q_init, xshift=4cm] {$q_2$};
    \node[state] (q2') [below=4cm of q2] {$q_2'$};
    \node[state] (q_accept) [below=4cm of q_init] {$q_\text{accept}$};
    \node[state] (q_reject) [below=4cm of q_accept] {$q_\text{reject}$};
    \node[state] (q3) [below=8cm of q_accept] {$q_3$};

    \path[->]
    (init) edge node {} (q_init)
    (q_init) edge node[sloped, above] {$0 \to 2, R$} (q1)
    (q_init) edge node[sloped, above] {$1 \to 2, R$} (q2)
    (q_init) edge node[sloped, above] {$\sqcup \to \sqcup, R$} (q_accept)

    (q_init) edge[loop, out=-15, in=15, distance=1cm] node[anchor=west] {$2\to 2, R$} (q_init)

    (q1) edge[loop, out=165, in=195, distance=1cm] node[anchor=east] {\shortstack{$0 \to 0, R$ \\ $1 \to 1, R$ \\ $2 \to 2, R$}}
    (q1)

    (q1) edge node[sloped, above] {$\# \to \#, R$} (q1')

    (q1') edge[loop, out=165, in=195, distance=1cm] node[anchor=east] {$2 \to 2, R$}
    (q1')
    (q1') edge node[sloped, above] {$1 \to 1, R$} (q_reject)
    (q1') edge node[sloped, above] {$0 \to 2, L$} (q3)

    (q2) edge[loop, out=-15, in=15, distance=1cm] node[anchor=west] {\shortstack{$0 \to 0, R$ \\ $1 \to 1, R$ \\ $2 \to 2, R$}}
    (q2)

    (q2) edge node[sloped, above] {$\# \to \#, R$} (q2')

    (q2') edge[loop, out=-15, in=15, distance=1cm] node[anchor=west] {$2 \to 2, R$}
    (q2')
    (q2') edge node[sloped, above] {$0 \to 0, R$} (q_reject)
    (q2') edge node[sloped, above] {$1 \to 2, L$} (q3)


    (q3) edge[loop, out=-105, in=-75, distance=1cm] node[anchor=north] {$\shortstack{$0 \to 0, L$ \\ $1 \to 1, L$ \\ $2 \to 2, L$ \\  $\# \to \#, L$}$} (q3)
    ;

    \draw[->, rounded corners=5pt]
    (q3) -- ($(q3)+(-7cm,0)$)
    -- node[sloped, above] {$\triangleright\to\triangleright, R$} ($(q_init)+(-7cm,0)$)
    -- (q_init);
  \end{tikzpicture}
\end{figure}

\section{Exercise 5.} Let $\L(M)$ be the language recognized by the TM $M$ i.e. the set of inputs $w\in\{0,1\}^*$ such that $M$ halts with output $1$ (on other inputs it may halt with output 0 or not halt). Consider the problem $E_\text{TM} = \{\langle M\rangle \,|\, M \text{ is a TM and } \L(M)=\varnothing\}$. Show that if $E_\text{TM}$ were decidable, then the halting problem would also be decidable. Conclude that $E_\text{TM}$ is undecidable.

For any encoding $\langle M,w\rangle$ of a TM and an input, construct a TM $M'$ such that
\begin{itemize}
  \item If $\forall x\in \{0,1\}^*, M'(x) = 1$ or $\L(M') = \{0,1\}^*$, then $M$ halts on $w$;
  \item If $\forall x\in \{0,1\}^*, M'(x)$ does not halt or $\L(M') = \varnothing$, then $M$ does not halt on $w$.
\end{itemize}

If $E_\text{TM}$ were decidable, then there would be a TM $N$ such that
$$N(\langle M'\rangle) = 1 \Leftrightarrow \L(M') = \varnothing \Leftrightarrow \text{ $M$ does not halt on $w$},$$
$$N(\langle M'\rangle) = 0 \Leftrightarrow \L(M') \neq \varnothing \Leftrightarrow \text{$M$ halts on $w$}.$$
So it would be possible to use $N$ to decide if a TM $M$ halts or not, or the halting problem would be decidable. But in fact, it is not. Thus, $E_\text{TM}$ is undecidable.
