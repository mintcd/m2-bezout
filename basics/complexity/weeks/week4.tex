\begin{center}
  {\bf
    \vspace{0.5cm}
    Complexity - Exercise Sheet 4\\
    \vspace{0.2cm}
    CHAU Dang Minh
  }
\end{center}

\section{Exercise 4.7.} What is wrong with the following proof of $\text{P}\neq \text{NP}$.

Assume that $\text{P}=\text{NP}$. Then there exists an algorithm $A$ and a polynomial $p(n)$ such that \textsc{SAT} is decided by $A$ in time $O(p(n))$. Assume that $p(n) = O(n^{37})$. By the Time Hierarchy Theorem, there exists a problem $P\in\text{DTIME}(n^{38})$ such that $P\notin\text{DTIME}(n^{37}\log{n^{37}}) = \text{DTIME}(n^{37}\log n)$. Since \textsc{SAT} is NP-complete, we can reduce $P$ to \textsc{SAT} and decided it in time $O(n^{37})$. But we have just shown that $P$ requires time $\omega(n^{37}\log n)$. This leads to a contradiction, hence the assumption $\text{P}=\text{NP}$ must be false.

\textit{Solution.} The assumption that $p(n) = O(n^{37})$ is not necessarily valid. But even if we make a weaker assumption that $p(n) = O(n^k)$ for some $k$, the proof is still flawed. The proof uses the Time Hierarchy Theorem to find a problem $P\in \text{DTIME}(n^{k+1})$ such that $P\notin \text{DTIME}(n^k)$, and arrive at a contradiction by reducing $P$ to \textsc{SAT} $O(n^k)$. However, suppose that $P$ is reduced to \textsc{SAT} in time $O(n^c)$ for some $c \ge 1$. Let $f$ be the reduction function. Then there is some $d\le c$ such that if for every $x\in\{0,1\}^*$, we have $|f(x)|\in O(n^d)$, because the length of the output cannot exceed the time of the reduction. Therefore, the time complexity to decide $P$ is in $O(n^c+n^{dk})$. For the contradiction to hold i.e. $O(n^c+n^{dk}) = O^{k}$, we must have $dk \le k$, or equivalently $d \le 1$, which is not provided by the proof.

\section{Exercise 4.12.} Consider the problem of determining whether a DNF formula $\phi$ has an equivalent formula having less than $k$ literals.
$$\textsc{Min-DNF} = \{\langle \phi,k\rangle \,|\, \exists \text{ DNF } \psi \text{ s.t. } \phi \equiv \psi \text{ and } \psi \text{ has } \le k \text{ occurrences of literals}\}.$$
Show that $\textsc{Min-DNF} \in \Sigma_2^p$.

\textit{Solution.} We measure the size of a DNF formula $\phi$, denoted by $|\phi|$, in terms of the number of literals it contains. Let $x=(x_1,\ldots,x_k), k\le |\phi|$  be the vector of variables in the formula $\phi$. We have $\langle \phi, k\rangle\in \textsc{Min-DNF}$ if and only if the following QBF is true.
$$\exists \psi  \,\,\forall x\,\, R(\phi,x),$$
where $R(\phi,x) = (\phi(x) = \psi(x))$. The domain of discourse for $\psi$ is the set of all DNF formulas with at most $k$ literals and the domain of discourse for $x$ is $\{0,1\}^k$. It is clear that the length of each $x$ is linear in $k$, and hence linear in $|\phi|$. Every formula $\psi$ in the former domain has at most $k$ variables, hence it has at most $2k$ different literals. We will also count $\wedge$ and $\vee$. Hence we need at most $\log(2k+2)$ bits to encode each literal and operator. The encoding first converts $\phi$ to the prefix notation, then encodes each literal/operator using $\log(2k+2)$ bits. Therefore, the size of the encoding of $\psi$ is at most $(2k-1)\log(2k+2)$ ($k$ literals and $k-1$ operators), which is polynomial in $|\phi|$. Thus, $P\in \Sigma_2^p$.

\section{Exercise 4.13.} The complexity class DP is defined as those decision problems that can be written as an intersection of an NP problem and a co-NP problem.

\begin{enumerate}[(a)]
  \item Show that $\textsc{3Sat-3Unsat} = \{\langle \phi,\psi\rangle \,|\, \phi\in\textsc{3Sat}, \psi\notin\textsc{3Sat}\}$ is \text{DP}-complete.
  \item Let $\alpha(G)$ be the independence number of a graph $G$. Let $\textsc{ExactIndSet} = \{\langle G,k\rangle \,|\, \alpha(G) = k\}$. Show that $\textsc{ExactIndSet}\in \text{DP}$.
  \item For two graphs $G_1=(V_1,E_1)$ and $G_2=(V_2,E_2)$, the lexicographic product of $G_1$ with $G_2$ is
        $$G = (V_1\times V_2, \{((u_1,u_2),(v_1,v_2)) \,|\, (u_1,v_1)\in E_1 \text{ or } u_1 = v_1 \text{ and } (u_2,v_2)\in E_2\}).$$
        Show that $\alpha(G)=\alpha(G_1)\cdot\alpha(G_2)$.
  \item Show that $\textsc{ExactIndSet}$ is DP-complete.
  \item Show that $\text{NP}\cup\text{coNP}\subseteq \text{DP}\subseteq \Sigma_2^p\cap \Pi_2^p$.
\end{enumerate}

\textit{Solution.}
\begin{enumerate}[(a)]
  \item We first show that $\textsc{3Sat-3Unsat}\in \text{DP}$. Let
        $$L_1 = \{\langle \phi,\psi\rangle \,|\, \phi\in\textsc{3Sat}\} \text{ and } L_2 = \{\langle \phi,\psi\rangle \,|\, \psi\notin\textsc{3Sat}\}.$$

        The problem $L_1$ is in NP, because we use a verifier for \textsc{3Sat} on the first component $\phi$ of every instance $\langle \phi,\psi\rangle\in L_1$. Similarly, $L_2^c = \{\langle \phi,\psi\rangle \,|\, \psi\in\textsc{3Sat}\} \in \text{NP}$, hence $L_2\in \text{coNP}$. We have $\textsc{3Sat-3Unsat} = L_1\cap L_2$, hence $\textsc{3Sat-3Unsat}\in \text{DP}$.

        Next, we show that $\textsc{3Sat-3Unsat}$ is DP-hard. In fact, $L_1$ is NP-complete because we can reduce every instance $\phi\in \textsc{3Sat}$ to $\langle \phi,\psi_0\rangle\in L_1$, where $\psi_0$ is a fixed formula. Similarly, $L_2^c$ is NP-complete. Therefore, for every co-NP problem $M$, we have
        $$x\in M \iff x \notin M^c \xLeftrightarrow{L_2^c\in \text{NP}} f(x)\notin L_2^c \iff f(x)\in L_2,$$
        where $f$ is a polynomial-time reduction from $M^c$ to $L_2^c$, or $L_2$ is co-NP-complete. Therefore, for every $M = M_1\cap M_2\in \text{DP}$, where $M_1\in \text{NP}$ and $M_2\in\text{co-NP}$, there exist polynomial-time reductions $u$ from $M_1$ to $L_1$ and $v$ from $M_2$ to $L_2$. We define the reduction $h$ from $M$ to $\textsc{3Sat-3Unsat}$ as
        $h(x) = \langle u(x),v(x)\rangle.$
        It is clear that $h$ is computable in polynomial time. Therefore, $\textsc{3Sat-3Unsat}$ is DP-complete.
  \item Let $L_1 = \{\langle G,k\rangle \,|\, \alpha(G)\ge k\}$ and $L_2 = \{\langle G,k\rangle \,|\, \alpha(G)\le k\}$. The problem $L_1$ is exactly our well-known \textsc{IndSet} problem, because $\langle G,k\rangle\in L_1$ if an only if $G$ has an independent set of size at least $k$. Hence $L_1$ is \text{NP}-complete. Using similar argument as in question (a), we derive that $L_2$ is co-NP-complete, because its complement is $\{\langle G,k\rangle \,|\, \alpha(G) \ge k+1\}$, a slightly modification of $L_1$, which is NP-complete. Therefore, $\textsc{ExactIndSet} = L_1\cap L_2\in \text{DP}$.
  \item Let $I_1\subseteq V_1$ and $I_2\subseteq V_2$ be two independent sets of $G_1$ and $G_2$ respectively. We show that $I = I_1\times I_2$ is an independent set of $G$. For every $(u_1,u_2),(v_1,v_2)\in I$, we have $u_1,v_1\in I_1$ and $u_2,v_2\in I_2$. Since $I_1$ and $I_2$ are independent sets, we have $(u_1,v_1)\notin E_1$ and $(u_2,v_2)\notin E_2$. Therefore, by the definition of the lexicographic product, $((u_1,u_2),(v_1,v_2))\notin E$. Hence, $I$ is an independent set of $G$. Therefore, if $I_1$ and $I_2$ are maximum independent sets of $G_1$ and $G_2$ respectively, then $I$ is an independent set of $G$ with size $|I| = |I_1|\cdot |I_2|$. This shows that $\alpha(G)\ge \alpha(G_1)\cdot \alpha(G_2)$.

        On the other hand, let $I\subseteq V$ be an independent set of $G$. Let $I_1 = \{u \,|\, (u,v)\in I\}$. We claim that $I_1$ is an independent set of $G_1$. If $|I_1| = 1$, we are done. Otherwise, for every $u,v\in I_1$, there exist $(u,u'),(v,v')\in I$ for some $u',v'\in V_2$. Since $I$ is an independent set of $G$, we have $((u,u'),(v,v'))\notin E$. By the definition of the lexicographic product, this implies that $(u,v)\notin E_1$. Hence, $I_1$ is an independent set of $G_1$. Similarly, $I_2 = \{v \,|\, (u,v)\in I\}$ is an independent set of $G_2$. Since $I\subseteq I_1\times I_2$, we have $|I|\le |I_1|\cdot |I_2|$. Therefore, if $I$ is a maximum independent set of $G$, then $\alpha(G) = |I|\le |I_1|\cdot |I_2|\le \alpha(G_1)\cdot \alpha(G_2)$. Combining this with the previous result, we have $\alpha(G) = \alpha(G_1)\cdot \alpha(G_2)$.
  \item Let $M\in \text{DP}$. Since $\textsc{3Sat-3Unsat}$ is DP-complete, we can reduce $M$ to $\textsc{3Sat-3Unsat}$ in polynomial time. The remaining is to reduce $\textsc{3Sat-3Unsat}$ to $\textsc{ExactIndSet}$. Let $\langle \phi,\psi\rangle$ be an instance of $\textsc{3Sat-3Unsat}$. Without loss of generality, assume that  both $\phi$ and $\psi$ have $n$ clauses (by adding true clause $(x\vee \neg x \vee y)$ to the formula having fewer clauses). For every formula $\phi$, we construct the graph $H(\phi)$ as in our previous proof for that \textsc{IndSet} is NP-complete.
        \begin{enumerate}[1.]
          \item For each clause $C_i = (l_{i1}\vee l_{i2}\vee l_{i3})$ in $\phi$, we create three vertices $v_{i1}, v_{i2}, v_{i3}$ corresponding to the three literals $l_{i1}, l_{i2}, l_{i3}$ and add edges between every pair of them.
          \item For every pair of vertices $v_{ij}$ and $v_{kl}$, where $i\neq k$, we add an edge $(v_{ij}, v_{kl})$ if and only if the literals $l_{ij}$ and $l_{kl}$ are complementary.
        \end{enumerate}

        If $\phi$ is satisfiable, we can select one true literal from each clause to form an independent set of size $n$. Suppose that there is an independent set of size $n+1$. Then there exist two vertices from the same set $\{l_{i1}, l_{i2}, l_{i3}\}$ for some $i$. But this contradicts the fact that they are all connected by edges. Hence, $\alpha(H(\phi)) = n$.

        If $\phi$ is not satisfiable, then for any selection of $n$ vertices, either there are two vertices corresponding to complementary literals, or there exists at least one clause $C_i$ such that none of its literals is selected. The first case contradicts the independence of the set, while the second case brings us back to the satisfiable case with $n-1$ clauses, which also raises a contradiction. Therefore, $\alpha(H(\phi)) \le n-1$.

        Next, for any two graphs $G_1 = (V_1,E_1)$ and $G_2 = (V_2,E_2)$, let $G_1\vee G_2$ by the graph obtained by collecting all vertices and edges of $G_1$ and $G_2$ and adding edges between every pair of vertices $u\in V_1$ and $v\in V_2$. It is clear that $\alpha(G_1\vee G_2) = \max(\alpha(G_1), \alpha(G_2))$, because any independent set of $G_1\vee G_2$ can only contain vertices from either $G_1$ or $G_2$. Also denote by $G_1\circ G_2$ the lexicographic product of $G_1$ and $G_2$.

        Let $E_n$ be the graph with $n$ isolated vertices. For every formula $\phi$, consider the graph
        $$G(\phi) = (H(\phi)\circ E_{n+1}) \vee E_{(n-1)(n+1)}.$$

        If $\phi\in \textsc{3Sat}$, then $\alpha(H(\phi)) = n$. Hence $\alpha(G(\phi)) = \max\{n(n+1), (n-1)(n+1)\} = n(n+1)$. If $\phi\notin \textsc{3Sat}$, then $\alpha(H(\phi)) \le n-1$. Hence $\alpha(G(\phi)) = (n-1)(n+1)$.

        Now for the instance $\langle \phi,\psi\rangle$ of $\textsc{3Sat-3Unsat}$, we construct the graph
        $$G = G(\phi)\circ G(\phi) \circ G(\psi).$$

        Consider four cases.

        \begin{enumerate}[1.]
          \item If $\phi\in \textsc{3Sat}$ and $\psi\in \textsc{3Sat}$, then $\alpha(G) = n^3(n+1)^3.$
          \item If $\phi\in \textsc{3Sat}$ and $\psi\notin \textsc{3Sat}$, then $\alpha(G) = (n-1)n^2(n+1)^3.$
          \item If $\phi\notin \textsc{3Sat}$ and $\psi\notin \textsc{3Sat}$, then $\alpha(G) = (n-1)^2n(n+1)^3.$
          \item If $\phi\notin \textsc{3Sat}$ and $\psi\in \textsc{3Sat}$, then $\alpha(G) = (n-1)^3(n+1)^3.$
        \end{enumerate}

        It is clear that the value in the second case is not equal to other values. Let $k = (n-1)n^2(n+1)^3$, we have $\langle \phi,\psi\rangle \in \textsc{3Sat-3Unsat} \iff \alpha(G) = k.$ Our constructs are all computable in polynomial time. Therefore, we have reduced $\textsc{3Sat-3Unsat}$ to $\textsc{ExactIndSet}$ in polynomial time, and hence $\textsc{ExactIndSet}$ is DP-complete.
  \item Note that $\{0,1\}^*$ and $\varnothing$ are in NP, since we can use the verifier that always accepts and rejects, respectively. Hence $\{0,1\}^*$ is in co-NP. Therefore for every $L\in \text{NP}$, we have $L = L\cap \{0,1\}^* \in \text{DP}.$ For every $L\in \text{co-NP}$, we have $L = \{0,1\}^* \cap L \in \text{DP}.$ This shows that $\text{NP}\cup \text{co-NP}\subseteq \text{DP}$.

        Next, let $L\in \text{DP}$. Then there exist $L_1\in \text{NP}$ and $L_2\in \text{co-NP}$ such that $L = L_1\cap L_2$. The corresponding quantified boolean formula is $\exists y_1\, R_1(x,y_1)$ and $\forall y_2\, R_2(x,y_2)$. Therefore, we can express $x\in L$ if and only if
        $$T = \exists y_1\, R_1(x,y_1) \wedge \forall y_2\, R_2(x,y_2)$$
        is true. Since $y_1$ does not appear in $R_2$ and $y_2$ does not appear in $R_1$, we can rewrite $T$ in two equivalent forms.
        $$T = \exists y_1\, \forall y_2\, R_1(x,y_1) \wedge R_2(x,y_2) =  \forall y_2\,\exists y_1\, R_1(x,y_1) \wedge R_2(x,y_2).$$

        Therefore, $x\in L$ if and only if the above QBFs are true. This shows that $L\in \Sigma_2^p \cap \Pi_2^p$. Thus, $\text{DP}\subseteq \Sigma_2^p \cap \Pi_2^p$.
\end{enumerate}
