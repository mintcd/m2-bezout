\section{Exercise 9.} Prove Lagrange's theorem. Deduce that a group of prime order is cyclic.

Let $G$ be a finite group and $H$ be a subgroup of $G$. Recall that the left coset of $H$ in $G$ with respect to an element $x \in G$ is the set $xH = \{xh : h \in H\}$. Suppose that there is $h_1, h_2\in H$ such that $xh_1 = xh_2$. Multiplying both sides on the left by $x^{-1}$, we get $h_1 = h_2$. So we must have that $\text{card}(xH) = \text{card}(H)$.

Since $G$ is finite, there are only finitely many distinct left cosets of $H$ in $G$. Let all of them be $x_1H, x_2H, \ldots, x_mH$, where $m\le\text{card}(G)$. We claim that

\begin{enumerate}
  \item The cosets are pairwise disjoint i.e. for every $i,j\in \{1,\ldots,m\}$ and $i\neq j$, we have $x_iH \cap x_j H = \varnothing$. Indeed, if there is $y\in x_iH \cap x_j H$, then $y\in x_iH$. Hence, there is $h\in H$ such that $y = x_ih$. Therefore, $yH = (x_ih)H = x_i(hH) = x_iH$. Similarly, $yH = x_jH$. So, $x_iH = x_jH$, contradicting the assumption that they are distinct.
  \item $G = x_1H \cup x_2H \cup \ldots \cup x_mH.$ Indeed, for any $x\in G$, there is $i\in\{1,\ldots,m\}$ such that $xH = x_iH$. If not, then $xH$ is a new left coset, contradicting the maximality of $m$.
\end{enumerate}

Therefore, we have $\text{card}(G) = \text{card}(x_1H) + \ldots + \text{card}(x_m H) = m \times \text{card}(G)$, or $\text{card}(H)$ divides $\text{card}(G)$.


To deduce that a group of prime order is cyclic, let $G$ be a group of prime order $p$. Let $x\in G$ and $x\neq e$. Since $e, a\in \langle x \rangle$, we have $\text{card}(\langle x \rangle) > 1$. By Lagrange's theorem, the order of $x$ divides the order of $G$. Since $p$ is prime, we must have $\text{card}(\langle x \rangle) = p = \text{card}(G)$. Hence, $\langle x \rangle = G$, or $G$ is cyclic.


\section{Exercise 11.} Prove that a subgroup $H$ of a group $G$ is normal if and only if for all $x\in G$ and $h\in H$ one has $xhx^{-1}\in H$ and also if and only if for all $x\in G$, $xHx^{-1} = H$.

Suppose that $H$ is a normal subgroup of $G$. Then, for every $x\in G$, we have $xH = Hx$. Therefore, for every $h\in H$, there is $h'\in H$ such that $xh = h'x$. Multiplying both sides on the right by $x^{-1}$, we get $xhx^{-1} = h' \in H$. Conversely, suppose that for every $x\in G$ and $h\in H$, we have $xhx^{-1}\in H$. Then, for every $x\in G$ and $h\in H$, there is $h'\in H$ such that $xh = h'x$. Therefore, $xH \subseteq Hx$. Similarly, we can show that $Hx \subseteq xH$. Hence, $xH = Hx$, or $H$ is a normal subgroup of $G$.

The other equivalence is proved as follows.
\begin{align*}
  \text{$H$ is normal} & \iff \forall x\in G, xH = Hx                  \\
                       & \iff \forall x\in G, xHx^{-1} = Hxx^{-1} = H.
\end{align*}

\section{Exercise 18.} \textbf{(Permutation Group)} Let $\S^n$ be the permutation group of the set $\{1,2,\ldots,n\}$.
\begin{enumerate}
  \item Show that for every $\sigma\in \S^n$ and every cycle $(i_1,\ldots,i_k)$ one has $\sigma(i_1,\ldots,i_k)\sigma^{-1} = (\sigma(i_1),\ldots,\sigma(i_k))$.
  \item Show that every element of $\S^3$ is a product of transpositions. Let $n\ge 2$ and $\sigma\in\S^n$. Show that if $\sigma(n)\ne n$, then there exists a transposition such that $\tau\circ \sigma(n) = n$. Conclude that for every $n\in\NN^*$, every element of $\S^n$ is a product of transpositions.
  \item Show that every $\sigma\in\S^n$ can be written as a product of cycles with disjoint supports.
  \item We want to show that for $n\ge 3$, $Z(\S^n) = \{I\}$. Let $\sigma\in Z(\S^n)$. Show that for every $i\ne j$ one has $(\sigma(i),\sigma(j)) = (i,j)$. Deduce that $\sigma = I$.
  \item Let us consider the subset $H$ of $\S^4$ defined by
        \[
          H = \{I, (12)(34), (13)(24), (14)(23)\}.
        \]
        Show that $H$ is an abelian normal subgroup of $\S^4$.
\end{enumerate}

\begin{enumerate}
  \item Let $c=(i_1,\ldots, i_k)$. For convenience, let $i_{k+1} = i_1$. Consider $x\in [n]$.
        \begin{itemize}
          \item If $x = \sigma(i_r)$ for some $r\in [k]$, then
                \[
                  \sigma c\sigma^{-1}(\sigma(i_j)) = \sigma c(i_j) = \sigma(i_{j+1}).
                \]
          \item If $x\notin \{\sigma(i_1),\ldots,\sigma(i_k)\}$, then $\sigma^{-1}(x)\notin \{i_1,\ldots,i_k\}$, so $c\sigma^{-1}(x) = \sigma^{-1}(x)$. Therefore,
                \[
                  \sigma c\sigma^{-1}(x) = \sigma(\sigma^{-1}(x)) = x.
                \]
        \end{itemize}

        Therefore, $\sigma(i_1,\ldots,i_k)\sigma^{-1} = (\sigma(i_1),\ldots,\sigma(i_k))$.
  \item We have $\S^3 = \{I, (12), (13), (23), (123), (132)\}$. The identity $I$ is the product of zero transpositions, or we may write differently as $I = (12)(12)$. Also, $(123) = (13)(12)$ and $(132) = (12)(13)$.

        Let $\tau = (\sigma(n),n)$, we have $(\tau\circ \sigma) (n) = \tau(\sigma(n)) = n$.

        Now we use induction to show that for every $n\in \NN^*$ every element of $\S^n$ is a product of transpositions. The base case $n=2$ is trivial. Suppose that the statement is true for some $n\ge 2$. Let $\sigma\in \S^{n+1}$. If $\sigma(n+1) = n+1$, then $\sigma\in \S^n$ and by the induction hypothesis, $\sigma$ is a product of transpositions. If $\sigma(n+1)\ne n+1$, then there exists a transposition $\tau$ such that $(\tau\circ \sigma)(n+1) = n+1$. Therefore, $\tau\circ \sigma\in \S^n$. By the induction hypothesis, $\tau\circ \sigma$ is a product of transpositions. Hence, $\sigma = \tau\circ (\tau\circ \sigma)$ is also a product of transpositions.
  \item Let $\sigma\in \S^n$. If $\sigma = I$, then we are done. Suppose that $\sigma\ne I$. Then, there is $i_1\in [n]$ such that $\sigma(i_1)\ne i_1$. Let $i_2 = \sigma(i_1)$. If $\sigma(i_2) = i_1$, then we have found a cycle $(i_1,i_2)$. Otherwise, let $i_3 = \sigma(i_2)$. If $\sigma(i_3) = i_1$, then we have found a cycle $(i_1,i_2,i_3)$. Otherwise, we continue this process. Since $[n]$ is finite, there must be $k\in \{2,\ldots,n\}$ such that $\sigma(i_k) = i_1$. Therefore, we have found a cycle $(i_1,i_2,\ldots,i_k)$.

        Now, let $\sigma' = (i_1,i_2,\ldots,i_k)^{-1}\circ \sigma$. We have $\sigma'(i_j) = i_j$ for every $j\in [k]$. If $\sigma' = I$, then we are done. Otherwise, we repeat the above process to find another cycle with disjoint support. Since $[n]$ is finite, this process must end after finitely many steps. Therefore, we can write $\sigma$ as a product of cycles with disjoint supports.

  \item Since $\sigma\in Z(\S^n)$, for every $i\ne j$, we have $(i,j)\sigma = \sigma(i,j)$. Multiplying both sides on the right by $\sigma^{-1}$, we get
        \[
          (i,j) = \sigma(i,j)\sigma^{-1} = (\sigma(i),\sigma(j)).
        \]
        Suppose that $\sigma \neq I$, or that there is $i\in [n]$ such that $\sigma(i) = j\ne i$. Since $n\ge 3$, there is $k\in [n]$ such that $k\ne i$ and $k\ne j$. Therefore, we have
        \[
          (i,k) = (\sigma(i),\sigma(k)) = (j,\sigma(k)).
        \]
        Hence $\{j,\sigma(k)\} = \{i,k\}$. But $j\ne i$ and $j\ne k$, which is a contradiction. Therefore $Z(S^n) = \{I\}$.
  \item We have
        $$((12)(34))^2 = I, (12)(34)(13)(24) = (14)(23), (12)(34)(14)(23) = (13)(24),$$
        and similarly for other two double transpositions. Therefore, $H$ is a subgroup of $\S^4$. From these equalities, we also have $(12)(34)(13)(24) = (13)(24)(12)(34)$ (equal to $(14)(23)$) and equalities of the same forms. Hence, $H$ is abelian. Finally, for every $\sigma\in \S^4$, by question 1, we have $$\sigma ((12)(34)) \sigma^{-1}  = \sigma ((12)\sigma^{-1}\sigma (34)) \sigma^{-1} =  (\sigma (12) \sigma^{-1})(\sigma (34) \sigma^{-1}) = (\sigma(1),\sigma(2))(\sigma(3),\sigma(4))\in H.$$
        Similarly for other two double transpositions. Therefore, $H$ is normal.
\end{enumerate}