\documentclass[11pt,a4paper]{article}

% Import common setup
% Common LaTeX setup for all exercises
% This file contains all shared packages and commands

% Packages
\usepackage[utf8]{inputenc}
\usepackage{amssymb}
\usepackage{amsmath}
\usepackage{graphicx}
\usepackage{amscd}
\usepackage{vmargin}
\usepackage{mathrsfs,color,comment}
\usepackage{multirow}
\usepackage{cases}
\usepackage{xfrac}
\usepackage{amsthm}
\usepackage{parskip}
\usepackage{mathtools}
\usepackage{xparse}
\usepackage{algorithm}
\usepackage[noend]{algpseudocode}
\usepackage{biblatex}
\usepackage{listings}
\usepackage{matlab-prettifier}
\usepackage{marvosym}
\usepackage{subcaption}
\usepackage{enumerate}
\usepackage{tikz}
\usepackage{titlesec}


% TikZ libraries
\usetikzlibrary{arrows.meta, automata,positioning, calc}

% Margins
\setmarginsrb{2.5cm}{1.3cm}{2.5cm}{1.3cm}{0cm}{0cm}{0cm}{0cm}

% Theorem environments
\newtheorem{lemma}{Lemma}
\newtheorem{proposition}{Proposition}
\newtheorem*{lemmas}{Lemmas}

% Custom commands
\newcommand\norm[1]{\left\lVert#1\right\rVert}
\newcommand\sym{\mathcal{S}}

\definecolor{myblue}{HTML}{4478a4}

\newcommand{\scrP}{\mathscr{P}}
\newcommand{\tr}{\mathrm{tr}}

% Blackboard bold commands
\newcommand{\QQ}{\mathbb{Q}}
\newcommand{\WW}{\mathbb{W}}
\newcommand{\EE}{\mathbb{E}}
\newcommand{\RR}{\mathbb{R}}
\newcommand{\TT}{\mathbb{T}}
\newcommand{\YY}{\mathbb{Y}}
\newcommand{\UU}{\mathbb{U}}
\newcommand{\II}{\mathbf{1}}
\newcommand{\OO}{\mathbb{O}}
\newcommand{\PP}{\mathbb{P}}
\renewcommand{\AA}{\mathbb{A}}
\renewcommand{\SS}{\mathbb{S}}
\newcommand{\DD}{\mathbb{D}}
\newcommand{\FF}{\mathbb{F}}
\newcommand{\GG}{\mathbb{G}}
\newcommand{\HH}{\mathbb{H}}
\newcommand{\JJ}{\mathbb{J}}
\newcommand{\KK}{\mathbb{K}}
\newcommand{\LL}{\mathbb{L}}
\newcommand{\ZZ}{\mathbb{Z}}
\newcommand{\XX}{\mathbb{X}}
\newcommand{\CC}{\mathbb{C}}
\newcommand{\VV}{\mathbb{V}}
\newcommand{\BB}{\mathbb{B}}
\newcommand{\NN}{\mathbb{N}}
\newcommand{\MM}{\mathbb{M}}

% Calligraphic commands
\newcommand{\A}{\mathcal{A}}
\newcommand{\B}{\mathcal{B}}
\newcommand{\C}{\mathcal{C}}
\newcommand{\D}{\mathcal{D}}
\newcommand{\E}{\mathcal{E}}
\newcommand{\F}{\mathcal{F}}
\newcommand{\G}{\mathcal{G}}
\renewcommand{\H}{\mathcal{H}}
\newcommand{\I}{\mathcal{I}}
\newcommand{\J}{\mathcal{J}}
\newcommand{\K}{\mathcal{K}}
\renewcommand{\L}{\mathcal{L}}
\newcommand{\M}{\mathcal{M}}
\newcommand{\N}{\mathcal{N}}
\renewcommand{\O}{\mathcal{O}}
\renewcommand{\P}{\mathcal{P}}
\newcommand{\Q}{\mathcal{Q}}
\newcommand{\R}{\mathcal{R}}
\renewcommand{\S}{\mathcal{S}}
\newcommand{\T}{\mathcal{T}}
\newcommand{\U}{\mathcal{U}}
\newcommand{\V}{\mathcal{V}}
\newcommand{\W}{\mathcal{W}}
\newcommand{\X}{\mathcal{X}}
\newcommand{\Y}{\mathcal{Y}}
\newcommand{\Z}{\mathcal{Z}}

% Script commands
\newcommand{\Ac}{\mathscr{A}}
\newcommand{\Bc}{\mathscr{B}}
\newcommand{\Cc}{\mathscr{C}}
\newcommand{\Dc}{\mathscr{D}}
\newcommand{\Ec}{\mathscr{E}}
\newcommand{\Fc}{\mathscr{F}}
\newcommand{\Gc}{\mathscr{G}}
\newcommand{\Hc}{\mathscr{H}}
\newcommand{\Ic}{\mathscr{I}}
\newcommand{\Jc}{\mathscr{J}}
\newcommand{\Kc}{\mathscr{K}}
\newcommand{\Lc}{\mathscr{L}}
\newcommand{\Mc}{\mathscr{M}}
\newcommand{\Nc}{\mathscr{N}}
\newcommand{\Oc}{\mathscr{O}}
\newcommand{\Pc}{\mathscr{P}}
\newcommand{\Qc}{\mathscr{Q}}
\newcommand{\Rc}{\mathscr{R}}
\newcommand{\Sc}{\mathscr{S}}
\newcommand{\Tc}{\mathscr{T}}
\newcommand{\Uc}{\mathscr{U}}
\newcommand{\Vc}{\mathscr{V}}
\newcommand{\Wc}{\mathscr{W}}
\newcommand{\Xc}{\mathscr{X}}
\newcommand{\Yc}{\mathscr{Y}}
\newcommand{\Zc}{\mathscr{Z}}

% Title formatting
% Make \section look like normal text
\titleformat{\section}[runin]   % "runin" = same line, no big break
{\normalfont\normalsize\bfseries} % style: normal text size + bold
{}                             % no number
{0pt}                          % no space before text
{}                             % code before the title

\titlespacing*{\section}
{0pt}{1em}{0.5em} % {left}{before}{after}
\titleformat{\subsection}[runin]{\bfseries}{}{}{}[]
\titleformat{\subsubsection}[runin]{\bfseries}{}{}{}[]

% Exercise counters and commands
\newcounter{exo}
\newcounter{parti}

\def\exercice%
{\stepcounter{exo}\noindent{\bf Exercise \arabic{parti}.\arabic{exo}.} \ }%
{\vspace{0,3cm}}

\def\partie%
{\stepcounter{parti} \noindent {\bf \arabic{parti}.} --- }%
{\vspace{0,3cm}}

\def\exercices%
{\noindent{\bf Exercises.} \ }%
{\vspace{0,3cm}}

% Additional math commands
\def\NN{{\mathbb{N}}}
\def\ZZ{{\mathbb{Z}}}
\def\QQ{{\mathbb{Q}}}
\def\RR{{\mathbb{R}}}
\def\CC{{\mathbb{C}}}

\newcommand{\rank}{\mathrm{rank}}
\newcommand{\im}{\mathrm{Im}}

\NewDocumentCommand{\myarrow}{sm}{
  \IfBooleanTF{#1}{
    \xrightarrow{#2}
  }{
    \xrightarrow{\mathmakebox[\minarrow]{#2}}
  }
}
\renewcommand{\arraystretch}{2}

% Page style and formatting
\pagestyle{empty}
\setlength{\parindent}{0em}

% Matrix environment customization
\makeatletter
\renewcommand*\env@matrix[1][*\c@MaxMatrixCols c]{%
  \hskip -\arraycolsep
  \let\@ifnextchar\new@ifnextchar
  \array{#1}}
\makeatother
\def\env@matrix{\hskip -\arraycolsep
  \let\@ifnextchar\new@ifnextchar
  \array{*\c@MaxMatrixCols c}}

% Algorithm customization
\makeatletter
% Reinsert missing \algbackskip
\def\algbackskip{\hskip-\ALG@thistlm}
\makeatother


% Bibliography setup
\addbibresource{refs.bib}

\newcommand{\divides}{\,|\,}
\renewcommand{\mod}{\text{ mod }}

\begin{document}

\thispagestyle{empty}

\begin{center}

  \textsc{Universit\'e Gustave Eiffel} \hfill  \,\\
  M2 -- Semester 1 -- 2025/2026 \hfill \, \\
  \bigskip

  %%TITRE
  {\bf
    \vspace{0.9cm}
    Programming\\
    \vspace{0.2cm}
    CHAU Dang Minh
  }
\end{center}
\bigskip

\vspace{0.5cm}

\section{Exercise.} \textbf{(Divisibility Problem)} We try to derive a divisibility rule for each integer $m$. That is, we find small $a,b$ such that for each integer $n = 10d + u$,
$$m \divides 10d+u \iff m \divides ad+ bu.$$
\begin{enumerate}
  \item Try to generate such rule for each integer $m$.
  \item Look at the results for inputs up to $200$. Give explanations.
\end{enumerate}

\textit{Solution.} We will find integers $a,b$ such that $0\le a\le 10$ and $1-m\le b\le m-1$. Since $d$ is much larger than $u$ most of the cases, it is reasonable use the following order: $(a_1,b_1) < (a_2,b_2)$ if $a_1 < a_2$ or ($a_1 = a_2$ and $b_1 < b_2$). The solution will be the smallest pair $(a,b)$ satisfying the condition. The equivalence
$$m\divides a_1d + b_1u \iff m\divides a_2d + b_2u$$
is maintained during the following transformations:
\begin{enumerate}
  \item $a_2 = a_1 + \ell m$ or $b_2 = b_1 + \ell m$ for some integer $\ell$.
  \item $a_2 = \ell a_1$ and $b_2 = \ell b_1$ for some integer $\ell$ coprime with $m$.
  \item $a_1  =\ell a_2$ and $b_1 = \ell b_2$ for some integer $\ell$ coprime with $m$.
\end{enumerate}
Define a sequence of transformations
$$(a_0,b_0) =(10,1), (a_1,b_1), (a_2,b_2), \ldots, (a_k,b_k) = (a,b).$$

Let $c = \gcd(m ,10) \in \{1,2,5,10\}$. We see that $c$ divides $a_i$ for each $i\in [0,k]$. We will prove that the minimal $a=c$ can be achieved i.e. there exists a sequence of transformations such that $a = c$.

\begin{lemma}
  \label{lemma:exist-x}
  There exists an integer $x$ such that $10x \equiv c \mod m$ and $\gcd(x,m) = 1$.
\end{lemma}

\begin{proof}
  Let $m = \prod_{\alpha \in I} p_\alpha^{e_\alpha}$ be the prime factorization of $m$. For each $\alpha \in I$, we have $10x \equiv c \mod p_\alpha^{e_\alpha}$. Consider the following cases.

  \begin{enumerate}
    \item If $p_\alpha = 2$, then
          $$10x \equiv c \mod p_\alpha^{e_\alpha} \iff 5x \equiv \dfrac{c}{2}\mod 2^{e_\alpha - 1} \iff 5x \equiv \dfrac{c}{2}\mod 2^{e_\alpha - 1},$$
          where the second equivalence is valid since $p_\alpha$ and $5$ are coprime. We also have $c\in\{2,10\}$, so $\dfrac{c}{2}\in \{1,5\}$ and thus $x$ is not divisible by $2$.
    \item If $p_\alpha = 5$, then
          $$10x \equiv c \mod p_\alpha^{e_\alpha} \iff 2x \equiv \dfrac{c}{5}\mod 5^{e_\alpha - 1} \iff x\equiv \dfrac{c}{5}\cdot 2^{-1} \mod 5^{e_\alpha - 1}.$$
          Again, the second equivalence is valid since $p_\alpha$ and $5$ are coprime. We also have $c\in\{5,10\}$, so $\dfrac{c}{5}\in \{1,2\}$ and thus $x$ is not divisible by $5$.
    \item Otherwise, $p_\alpha$ is coprime with $10$. We have
          $$10x \equiv c \mod p_\alpha^{e_\alpha} \iff x \equiv 10^{-1}c \mod p_\alpha^{e_\alpha}.$$

          We also have $c\in\{1,2,5,10\}$, which is all coprime with $p_\alpha$. Thus, $x$ is not divisible by $p_\alpha$.
  \end{enumerate}

  By the Chinese Remainder Theorem, there exists an integer $x$ satisfying one of the congruences for $(\alpha_i)_{i\in I}$. We have $\gcd(x,m) = 1$ since for each prime factor $p_\alpha$ of $m$, $x$ is not divisible by $p_\alpha$.
\end{proof}

By the lemma, there exists an integer $x$ such that $10x \equiv c \mod m$ and $\gcd(x,m) = 1$. We have the following sequence of transformations:
$$(10,1) \longrightarrow (10x,x) \longrightarrow (c, x) \longrightarrow (c, x \mod m) \longrightarrow (c, x\mod m - m).$$

Now we will construct such an $x$. A straightforward way is to find $x_{p_i}$ satisfying $10x_{p_i} \equiv c \mod p_i^{e_i}$ for each prime factor $p_i$ of $m$, then use the Chinese Remainder Theorem to find $x$. However, it is not efficient since the factorization of $m$ is required. It is surprising that using the Extended Euclidean algorithm (EEA) is sufficient. We learn from the proof of Lemma~\ref{lemma:exist-x} to prove our following results.

\begin{lemma}
  \label{lemma:find-x-in-4}
  Let $x_0, y_0$ be integers such that $10x_0 + my_0 = c$. Let $s=\dfrac{m}{c}$. Then the set $$\left\{x_0, x_0 + s, x_0 + 2s, x_0 + 3s\right\}$$ contains an integer $x$ such that $\gcd(x,m) = 1$.
\end{lemma}

\begin{proof}
  Let $n = 2^{\alpha}5^{\beta}r$. Then
  $$c = \gcd(m,10) = 2^{\min(\alpha,1)}5^{\min(\beta,1)} \text{ and } s = 2^{\alpha - \min(\alpha,1)}5^{\beta - \min(\beta,1)}r.$$

  Consider a prime factor $p$ of $m$.
  \begin{enumerate}
    \item If $p=2$ i.e. $\alpha\ge 1$. We have two cases.
          \begin{itemize}
            \item $\alpha = 1$. Then $s = 5^{\beta - \min(\beta,1)}r$ is odd. Thus, there are exactly two odd numbers and two even numbers in the set $\{x_0, x_0 + s, x_0 + 2s, x_0 + 3s\}$.
            \item $\alpha \ge 2$. Then $5x_0 \equiv c \mod 2^{\alpha - 1}$ and $x_0$ is odd. Also, $s = 2^{\alpha - 1}5^{\beta - \min(\beta,1)}r$ is even. Thus, all numbers in the set $\{x_0, x_0 + s, x_0 + 2s, x_0 + 3s\}$ are odd.
          \end{itemize}
    \item If $p=5$ i.e. $\beta\ge 1$. We have two cases.
          \begin{itemize}
            \item $\beta = 1$. Then $s = 2^{\alpha - \min(\alpha,1)}r$ is not divisible by $5$. Checking all possible cases of $x_0$ and $s$ in modulo $5$, there are at least \textit{three} numbers that are not divisible by $5$.
            \item $\beta \ge 2$. Then $2x_0 \equiv \dfrac{c}{5} \mod 5^{\beta - 1}$ and $x_0$ is not divisible by $5$. Also, $s = 2^{\alpha - \min(\alpha,1)}5^{\beta - 1}r$ is divisible by $5$. Thus, all numbers in the set $\{x_0, x_0 + s, x_0 + 2s, x_0 + 3s\}$ are not divisible by $5$.
          \end{itemize}
    \item Otherwise, $p$ is coprime with $10$. We have $x_0 \equiv 10^{-1}c \mod p^{e}$. Since $c\in \{1,2,5,10\}$, $s$ is divisible by $p$. Thus, all numbers in the set $\{x_0, x_0 + s, x_0 + 2s, x_0 + 3s\}$ are not divisible by $p$.
  \end{enumerate}
  Therefore, we have to consider four cases for $\alpha$ and $\beta$. Each guarantees at least one number in the set $\{x_0, x_0 + s, x_0 + 2s, x_0 + 3s\}$ is not divisible by either $2$ or $5$. In fact, only in the case that $\alpha=\beta=1$ do we have to use the pigeonhole principle.
\end{proof}

From the proof of Lemma~\ref{lemma:find-x-in-4}, an integer $x$ in the set satisfies if $x$ is not divisible by $2$ or $5$. Moreover, the only edge case is when $\alpha=\beta=1$ i.e. $m=10r$ where $r$ is coprime with $10$. But in such cases, EEA returns $x_0=1$, which is automatically coprime with $m$. Therefore, we have a stronger result although there is almost no difference in practice.

\begin{lemma}
  Let $x_0, y_0$ be integers such that $10x_0 + my_0 = c$ and $x_0$ is returned by EEA. Let $s=\dfrac{m}{c}$. Then the set $$\left\{x_0, x_0 + s, x_0 + 2s\right\}$$ contains an integer $x$ such that $\gcd(x,m) = 1$.
\end{lemma}

Thus, the algorithm to find a divisibility rule for $m$ is as follows.

\begin{algorithm}
  \begin{algorithmic}
    \State $(x_0,y_0,c)\leftarrow \text{EEA}(m,10)$, $s \leftarrow m/c$.
    \If{$2 \divides x_0$}

    \Return $(c, (x_0 + s) \mod m - m)$.

    \Else

    \If{{$5 \divides x_0$}}

    \Return $(c, (x_0 + 2s) \mod m - m)$.

    \Else

    \Return $(c, x_0 \mod m - m)$.

    \EndIf
    \EndIf
  \end{algorithmic}
\end{algorithm}

\begin{proposition}
  The above algorithm returns the optimal divisibility rule for each integer $m$ in $O(1)$.
\end{proposition}
\begin{proof}
  The correctness of the algorithm is guaranteed by the previous lemmas. The time complexity is $O(\log(\min(m,10))) = O(1)$ since $10$ is a constant.
\end{proof}

\end{document}